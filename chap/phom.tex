% Fichier intropm.tex
\ProvidesFile{phom.tex}[version 2018-01-27]
% Decommenter les les lignes avec " %>> " pour tester le fichier
%


\chapter{Phonologie Minimaliste}	
  \sommaire
  
  \section{Bases th\'eoriques de la Phonologie Minimaliste}
    Que faut-il pour faire d'une th\'eorie/analyse phonologique, une th\'eorie/analyse minimaliste ? C'est ce \`a quoi nous allons tenter de r\'epondre dans cet atelier. Tout en faisant une synth\`ese des travaux concernant la possible architecture du composant phonologique -- dor\'enavant \newcommand\lphi{$ \Phi $} \lphi\ -- nous allons dans les pages qui suivent essayer d'\'ebaucher une phonologie minimaliste. Notre point de depart sera le mod\`ele de \cite{samuels2009structure,samuels2011architecture}.
    \subsection{Introduction}
    \marnote{\color{rosepal} Cette section contient du contenu adapt\'e de \cite{martins2012review-samuels}} 
    La phonologie minimaliste peut \^etre vue comme un programme de recherche qui vise \`a d\'ecrire le composant phonologique du langage. Selon \cite{samuels2009structure}, une phonologie minimaliste devrait \^etre fond\'ee sur trois piliers : (i) le Minimalisme, (ii) l'approche \textit{substance free} et (iii) la Phonologie \'Evolutionnaire. 
    
    La place du minimalisme (i.e. l'ensemble des lignes directrices qui constituent le programme minimaliste) dans une th\'eorie comme celle que nous tenterons d'\'ebaucher est n'est pas \`a discuter -- puisque cette th\'eorie se veut minimaliste par définition. 
    Ce que \cite{samuels2009structure} tente de faire c'est de formuler pour la phonologie ce qui a \'et\'e formul\'e pour le langage en g\'en\'eral, cf. Chomsky 2007 : 
    \begin{quote}
    \itshape	Quel minimum (i.e. plus petit ensemble contenant des traits et des op\'erations) peut \^etre attribu\'e \`a GU tout en gardant une puissance explicative suffisante pour rendre compte de tous les ph\'enom\`enes phonologiques attest\'es
    \end{quote}
\newcommand\sfree{\textit{substance free}}
    Du coup, la t\^ache de Samuels est double :
    \begin{itemize}
    	\item d\'eterminer ce qui est propre au composant phonologique de GU -- qui tout comme GU elle-m\^eme doit \^etre simple, dans le meilleur des cas -- en utilisant l'approche 
    	 \sfree\ comme levier.
    	\item attribuer le reste \`a des facteurs extra-linguistiques, essentiellement phonologiques, avec l'aide du mod\`ele \'evolutionniste de la phonologie \cite[cf.][]{blevins2004evolutionary}.
    \end{itemize}    
    
     
    \subsection[L'approche \textit{substance free}]{L'approche substance free}
      L'approche \sfree\ de la phonologie [Hale et Reiss 2000, Blaho 2008, \textit{inter alia}] repose essentiellement sur deux postulats :
	  \begin{enumerate}
	    \item le concept de \textit{phonologie} d\'esigne un syst\`eme abstrait qui gouverne le signifiant -- le niveau non significatif de la comp\'etence linguistique.
	    \item les primitives phonologiques sont \sfree\ : leur interpr\'etation [phon\'etique] est invisible pour la phonologie et, par cons\'equent, ne joue aucun r\^ole dans la computation phonologique. 
	  \end{enumerate}	
	  Hors, \`a partir du moment o\`u on fait abstraction du contenu phon\'etique rattach\'e aux traits phonologiques, il ne reste plus que des symboles. Il faut donc penser que le composant phonologique \lphi\ op\`ere sur des symboles abstraits appel\'es traits, cf. \cite{samuels2009structure}.
	  \begin{prat}
	  	\begin{itemize}
	  		\item On se fonde sur SPE [The \textit{Sound Patterns of English}]
	  		\item Les sch\`emes phonologiques sont les indices avec lesquels on travaille -- \important{on consid\`ere la langue dans son ensemble}.
	  	\end{itemize}
	  \end{prat}	 
	\subsection{La Phonologie Evolutionnaire}
	  \marnote{Cette section est tir\'ee de \cite{akpoue2016}, pp. 29--30.}
	  L'une des hypoth\`eses les plus fondamentales en mati\`ere de changement phon\'etique est celle qui r\'esulte des travaux des N\'eogrammairiens -- la r\'egularit\'e du changement phon\'etique -- et qui peut \^etre \'enonc\'ee comme suit :
	  
	  \ex. \textit{L'hypoth\`ese de la r\'egularit\'e du changement phon\'tique}\\
	  Si un son $ \alpha $ d'une langue donn\'ee devient $ \beta $ dans une langue s\oe ur ou un \'etat ult\'erieur de la m\^eme langue, $ \alpha $ deviendra toujours $ \beta $ dans cette langue s\oe ur ou cet \'etat de langue pour autant que les conditions qui ont favoris\'e sa mutation soient toutes remplies.

	  Toutefois, comme le note \cite{blevins2004evolutionary}, la t\^ache du phonologue est double : (i) mettre en \'evidence les sch\`emes auxquels ob\'eissent les `sons' (ii) proposer une explication. La Phonologie Evolutionnaire se donne pour mission la t\^ache (ii). 
	  Ainsi \cite{blevins2004evolutionary} va dresser une typologie identifiant trois facteurs fondamentaux de changement phon\'etique : \textsc{changement}, \textsc{hasard} et \textsc{choix} (cf. \Next).
	  \ex.	Typologie g\'en\'erale du changement phon\'etique, L = Locuteur, I = Interlocuteur
	  \a. \textsc{changement} : Le signal phon\'etique est mal per\c cu par l'interlocuteur \`a cause de : similitudes acoustiques entre l'\'enonc\'e tel que produit et l'\'enonc\'e tel que per\c cu\footnote{Voir Louria ... en ce qui concerne la reconnaissance des phon\`emes}; et des biais provenant du syst\`eme perceptuel humain.\\
	  L dit [anpa] I entend [ampa]
	  \b. \textsc{hasard} : Le signal phon\'etique est fid\`element per\c cu par l'interlocuteur mais il est  intrins\`equement phonologiquement ambigu. L'interlocuteur associe \`a l'\'enonc\'e une forme phonologique qui diff\`e de la forme phono\-logique pr\'esente dans la grammaire [langage-i] du locuteur.\\
	  L dit [\cdgl\nas{a}\cdgl] pour /\nas{a}\cdgl/ I entend [\cdgl\nas{a}\cdgl], et pense /\cdgl a/
	  \c. \textsc{choix} : Plusieurs variantes phon\'etique d'une forme phonologique uni\-que sont fid\`element per\c cus par l'interlocuteur. L'interlocuteur (a) acquiert un prototype ou exemple optimal qui diff\`ere de celui du locuteur ; et/ou (b) associe une forme phonologique \`a l'ensemble des variantes, forme phonologique diff\'erente de celle pr\'esente dans la grammaire du locuteur.\\ 
	  L dit [tu\cdgl\schwa la\Ng], [tu\cdgl\schwa la\Ng], [tu\cdgl la\Ng] for /tu\cdgl\schwa la\Ng/
	  I entend [tu\cdgl\schwa la\Ng], [tu\cdgl\schwa la\Ng], [tu\cdgl la\Ng], and assumes /tu\cdgl la\Ng/
	  \hfill (Adapt\'e de Blevins 2006 : 126)
	  
	  En r\'esum\'e, les changements phon\'etiques sont dus soit \`a une mauvaise perception du signal acoustique (\textsc{changement}), soit encore \`a une ambigu\"it\'e phonologique du signal per\c cu (\textsc{hasard}), soit enfin \`a un \textsc{choix} op\'er\'e par l'auditeur entre plusieurs options. 
	  
	\subsection{Dans GU, hors de GU}
	  \subsubsection{Le marquage : hors de GU}
	    Le marquage est une notion assez vague quoique centrale en linguistique moderne. Il est courant, dans les formulations, d'opposer formes marqu\'ees et formes non-marqu\'ees. Et on explique ce contraste en faisant appel \`a des principes tels que la complexit\'e. Voyons un ph\'enom\`ene assez familier : l'\'epenth\`ese.
	    \paragraph{L'\'epenth\`ese: une violation flagrante de la condition d'inclusion}
	    
	    \begin{defin}[Condition d'Inclusion]
	      L'output d'une opération ne doit contenir rien d'autre que son input.
	    \end{defin}
	    \ex. Francais \\
	    Input: \{/a/\{/il/,/a/\}\}
	    Output: [aTil]
	    	    
	     
	    L'\'epenth\`ese est purement phon\'etique i.e. cr\'e\'e par SM \cite[voir][]{samuels2009structure}.
	  \subsubsection{OCP : hors de GU}
	  OCP a \'et\'e disqualifi\'e pour deux raisons : 
	  \begin{itemize}
	  	\item il admet des exceptions, cf. OT
	  	\item il est formul\'e en fonction des syllabes -- qui sont elles-m\^emes disqualifi\'ees, voir plus bas.
	  \end{itemize}
      Il en va de m\^eme pour tous les universaux typologiques.
	  \subsubsection[Inn\'eit\'e des traits phonologiques]{Les traits : dans GU ... hors de GU}
	    
	    \begin{itemize}
	      \item Reiss, Halle, Chomsky 2007: oui. Reiss \& Halle : un ensemble de traits universels innés. Et on perd ceux au'on utilise pas.
	      \item Blaho, Samuels, Dresher : non. Les traits sont appris. Mais s'ils sont appris, comment les symboles sont-ils cr\'e\'es ?
	    \end{itemize} 
	\subsubsection{Le\c{c}on : phon\'etique n'est pas phonologique}
	  Ce qui est phon\'etique n'est pas phonologique
	  i.e. ce qui peut \^etre expliqu\'e par la phon\'etique n'est pas phonologique
	  e.g., OCP. mais aussi le principe de complexit\'e. Si un son change par ce principe, ce son reste phonologiquement intact.
	  
	  Si nous avons raison, alors le changement phon\'etique est invisible pour la phonologie i.e. les repr\'esentations phonologiques r\'esistent au changement phon\'etique. En avons-nous des preuves empiriques ?
	  \begin{itemize}
	  	\item \textit{L'\'epenth\`ese en Fran\c{c}ais}\\
	  	En r\'ealit\'e, il est th\'eoriquement possible d'\'eviter la violation de la condition d'inclu\-sion dans le cas de l'\'epenth\`ese en postulant par exemple que les formes vocaliques que prend l'auxiliaire \textit{avoir} \`a la troisi\`eme personne ont un /t/ qui n'est pas prononc\'e. Le sch\`eme /at/ $ \rightarrow $ [a] est une question de changement phon\'etique. 
	  	\item \textit{Le b implosif dans les langues Kwa} \cite[cf.][]{bogny2014}.
	  \end{itemize}
	  Conclusion : \motcle{le changement phon\'etique est invisible pour la phonologie.}
	  
	   
	  \begin{appli}
	  \textit{Harmonie vocalique}\\
	    Si une langue pr\'esente des vestiges d'harmonie vocalique et si RIEN D'AUTRE ne peut expliquer ces faits, alors il faudrait conclure que cette langue est une langue \`a harmonie vocalique. Le fait que cela ne se voit pas est dû au changement phon\'etique.
	  \end{appli} 
  \section{Les objets phonologiques}
	\begin{itemize}
		\item Les primitives phonologiques : des \alert{traits binaires} \cite[cf.][]{samuels2009structure,dresher2014arc}
		\item ce sont des r\'ealit\'es mentales et donc non pronon\c cables.
		\item ces traits `s'agglutinent' pour donner des \alert{phon\`emes}.
		\item les phon\`emes ne sont donc pas des pronon\c cables eux aussi.
		\item Selon \cite{dresher2014arc}, la t\^ache de l'enfant lors de l'acquisition du langage est d'arriver \`a un \textit{syst\`eme de traits} qui rende compte du contraste entre les phon\`emes de sa langue.
		\item ce \textit{syst\`eme de traits} correspond \`a ce qu'elle appelle \alert{hi\'erarchie contrastive}.
		\item une hierarchie contrastive se presente sous la forme d'un diagramme arborescent dont les n\oe uds sont binaires tr\`es souvent. 
		\item p.ex. \ex. \Tree [ \trait{+bas}  [.\trait{-bas} \trait{+haut} [.\trait{-haut} [.\trait{+post} \trait{+ATR}  \trait{-ATR} ] [.\trait{-post} \trait{+ATR}  \trait{-ATR} ] ] ] ]
		
		\item techniquement, chez Dresher, le terme \textit{hierarche contrastive} renvoi \`a un ensemble ordonn\'e de traits contrastif.  
	\end{itemize}
    \begin{defin}[Hierarchie contrastive]\label{def:herarchie_contrastive}
    	Pour toute langue L,  la hierarchie contrastive $ HC_L $ de L est l'ensemble ordonn\'e de tous les traits contrastifs de L.
    	
    	Formellement, on a $ HC_L = TC_L^> $ i.e. $ HC_L $ est \'egal \`a l'ensemble des traits constrastifs de L ($ TC_L $) muni d'une relation d'ordre total $ > $\footnote{`$ > $' signifie ici \textit{est plus haut que}}. 
    \end{defin}
    
    \begin{defin}[Trait contrastif]\label{def:trait contrastif}
    	Pour tout trait \trait{x} et toute langue L, \trait{x} est un trait contrastif de L ssi il permet de diff\'erentier au moins deux phon\`emes de L.
    \end{defin}
     
    \begin{itemize}
    	\item p.ex. la hierarchie en \Last\ peut \^etre reformul\'ee en \Next.
    	\ex. \trait{bas} $ > $ \trait{haut} $ > $ \trait{post} $ > $ \trait{ATR}.
    	
    	\item en effet, cet ordre contraint le nombre de diagrammes arborescents que l'on peut proposer pour rendre compte de la hierarchie contrastive dans cette langue. p.ex. le trait \trait{haut} ne peut \^etre une fille du trait \trait{ATR} selon cet ordre.  
    	\item la hi\'erarchie contrastive est un outil utilis\'e pour \'etudier le syst\`eme phonologique d'une langue.
    	\item Il d\'ecoule de la D\'efinition \ref{def:herarchie_contrastive} que la t\^ache du phonologue, dans ce domaine, est de :
    	\begin{enumerate}
    		\item identifier tous les traits contrastifs de la langue qu'il \'etudie.
    		\item trouver l'ordre dans lequel il apparaissent dans la hierarchie contrastive de cette langue.
    	\end{enumerate}
        \item comment le faire ?
        \item Les paires minimales et la m\'ethode distributionnelle ? 
        Inutiles pour nous car, dans le meilleur des cas, elles permettent juste de trouver les phon\`emes, pas les traits. 
        \item quelle solution alors ?
        \item on regarde l'activit\'e phonologique.
        \item En effet, Dresher assume que tous les traits contrastifs dans L sont aussi des traits actifs dans L et tous les traits actifs dans L sont aussi des traits contrastifs dans L
         
    \end{itemize}
    	
    \begin{defin}[Trait actif]\label{def:trait actif}
    	Pour tout trait \trait{x} et toute langue L, \trait{x} est un trait actif dans L ssi il est n\'ecessaire pour la formulation d'au moins une g\'en\'eralisation concernant la phonologie de L.
    \end{defin}
    \begin{itemize}
    	\item p.ex.
    	\begin{itemize}
    		\item \trait{rond} est actif dans une langue qui admet une harmonie d'arrondissement.
    		\item \trait{voix} est actif dans une langue qui admet une r\`egle de voisement.
    	\end{itemize}
    \end{itemize}
    
    \begin{comment}
	\begin{defin}[PHON-L]
		Pour toute langue L,  PHON-L est le plus petit ensemble contenant tous les traits phonologiques actifs dans L. Plus formellement PHON-L = $ \{ X : X \in PHON \land X\mbox{ est actif dans L}.\}$
	\end{defin}
	\end{comment}
	
	
  \section{Repr\'esentations et op\'erations basiques}
    \subsection{La repr\'esentations des phon\`emes}
      Les phon\`emes sont des ensembles de traits. Toute notation qui les repr\'esente comme tel est bonne \`a prendre. Ainsi le phon\`eme : \matriss{ht:--, bas:$ + $, nas:+} peut aussi \^etre not\'e comme en \Next.
      \ex. \begin{matrice}
      	\traitm{ht:$ - $},\\
      	\traitm{bas:$ + $},\\
      	\traitm{nas:$ + $}
      \end{matrice}
      
      Toutefois, par convenance, il est admis de pouvoir abr\'eger les phon\`emes par des symboles de l'API ou de l'alphabet latin en majuscules -- pour les phon\`emes consid\'er\'es comme ``archi-phon\`emes''.
      \ex. \begin{matrice}
      	\traitm{ht:$ - $},\\
      	\traitm{bas:$ + $},\\
      	\traitm{nas:$ + $}
      \end{matrice} :=\quad \nas{a}
         
    \subsection{Les repr\'esentations des s\'equences phonologiques}
      \subsubsection{Les syllabes, hors de GU}
        \cite{samuels2007stringtheory,samuels2009structure,samuels2011architecture} soutient que les syllabes ne sont pas des r\'ealit\'es grammaticales et que les repr\'esentations prosodiques sous forme de diagrammes arborescents devraient \^etre abandonn\'ees au profit de representations lin\'eaires. Son argumentation s'articule autour des points suivants :  
        \begin{itemize}
        	\item les sous-constituents syllabiques [cf. Phonologie du Gouvernement e.g., KLV] nous dispense de la n\'ecessit\'e d'un n\oe ud syllabique.
        	\item les travaux en psychologie ont conduit \`a l'id\'ee que les syllabes ne sont pas des r\'ealit\'e grammaticales, elles appartiennent au langage au sens large.
        	\item de plus, les syllabes ne sont pas des syntagmes et donc ne doivent pas etre represent\'ees comme tel. 
        	\item enfin, les r\`egles formul\'ees en faisant appel aux syllabes peuvent l'\^etre sans l'aide des syllabes.
        \end{itemize}
      \subsubsection{Repr\'esentation lin\'eaire} 
        \begin{itemize}
        	\item Une relation basique : $ \rightarrow $
        	\item X \fleche\ Y $ \equiv $\footnote{i.e. ``est identique \`a''} \dpr{X}{Y} \tequiv\ X $ \prec $ Y i.e. X \textit{pr\'ec\`ede} Y.
        	\item p. ex. /palp/ : \deb\ \fleche\ p \fleche\ a \fleche\ l \fleche\ p \fleche\ \fin\ \tequiv\ \dpr{\deb}{p}, \dpr{p}{a}, \dpr{a}{l}, \dpr{l}{p}, \dpr{p}{\fin}	
        	\item les affixes ont une presentation phonologique en termes de s\'equence \`a laquelle il manque un des symboles terminaux. plus concretement, ils ont une variable \var[i] \`a la place de \deb\ ou \fin.
        	\item p. ex. /-abl/ : \var[1] \fleche\ a \fleche\ b \fleche\ l \fleche\ \fin\ \tequiv\ \dpr{\var[i]}{a}, \dpr{b}{l}, \dpr{l}{\fin} 
        	
        \end{itemize}
      \subsubsection{Representations sinuso\"idales}
        Ce sont des representations en forme de courbe qui sont cens\'ees saisir la hierarchie de sonorit\'e dans une s\'equence de sons.
    \subsection{Les op\'erations basiques}
      \subsubsection{Chercher}
        
      \subsubsection{Copier}
        
      \subsubsection{Supprimer}
           	
  \section{L'interface Syntaxe-Phonologie}
    \subsection{Lin\'earisation et externalisation des copies}
      \subsubsection{Lin\'earisation des expressions linguistiques}
        \paragraph{La n\'ecessit\'e de lin\'eariser les structures syntaxiques}
        
        \paragraph{L'axiome de Correspondance lin\'eaire}
          
        \paragraph{Le param\`etre de t\^ete}
          
      \subsubsection{Externalisation des copies}
    \subsection[D\'erivation phonologique par phases]{Th\'eorie des phases et localit\'e des processus phonologiques}
  \section{Construire une phonologie minimaliste autosegmentale}
        
\endinput 