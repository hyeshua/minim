% Fichier semin.tex
%

\ProvidesFile{lexmo.tex}[version 2018-02-14]
% Decommenter les les lignes avec " %>> " pour tester le fichier
%
% ===================================
%% File partly assisted by LyX 2.2.2 
%% LyX tasks were :
%%	- document arborescence


%\documentclass[oneside,11pt]{book}
%\usepackage[T1]{fontenc}
%\usepackage[latin1]{inputenc}


%\begin{document}

\chapter{Morphologie et lexique}

  \sommaire


\section{Introduction}
  \subsection[Un composant lexical/morphologique dans GU ?]{Lexique/morphologie : un composant autonome de GU ?}
    \subsubsection{Au niveau s\'emantique}
    
    \subsubsection{Au niveau syntaxique}
    
    \subsubsection{Contre un lexique actif}
    La morpologie, c'est de la syntaxe
  \subsection{Cons\'equences th\'eoriques}
    Trois problemes a resoudre cependant :

    Qu'est-ce qu'un morph\`eme ? Rendre compte de la variation phonologique
    liee a l'accord

    La structure interne des entrees lexicales : on ne parle de composition,
    mais bien de structure. les entrees lexicales apparaissent comme des
    objets syntaxiques complexes deja. Cela vaut aussi pour les categories
    fonctionnelles. comment integrer ca a la theorie ?

Et le lexique dans tout ca ?

nous allons explorer deux ecoles et essayer de voir comment on peut
repondre a ces trois questions tout en accord avec les lignes directrices
du PM

  \section{La Morphologie Distribu\'ee}
    
    \subsection{Vue d'ensemble}
      \subsubsection{Qu'est-ce que la Morphologie Distribu\'ee ?}
      La Morphologie Distribu\'ee (en abr\'eg\'e DM d'apr\`es les initiales de l'Anglais \textit{Distributed Morphology}) est une th\'eorie sur l'architecture de la grammaire qui réfute l'hypoth\`ese lexicaliste cad l'id\'ee qu'il puisse exister une compétence qui permette de construire les mots qui soit différente de celle qui construit les \'enonc\'es. Heidi Harley le r\'esume de fort belle mani\`ere :
      \begin{quote}
    	Plusieurs th\'eories assument qu'il existe un 'lexique g\'en\'eratif', un algorithme\footnote{le mot `algorithme', pour ce qui nous concerne ici, d\'esigne une compétence intuitive qui fait typiquement partie de la comp\'etence linguistique.} qui construit les mots que la syntaxe utilise pour construire les phrases. Dans ces th\'eories, le langage tire sa fameuse capacit\'e g\'en\'erative\footnote{Ce qui fait la particularit\'e du langage humain, c'est le fait qu'il permet, \`a partir d'un nombre fini d'\'el\'ements, de g\'en\'erer des \'enonc\'es en taille et en nombre infini. Cette propri\'et\'e a d'abord \'et\'e reconnue par des philosophes tels que Descartes, voir Fitch Hauser et Chomsky 2002 et Chomsky 2013 pour une discussion récente.} \`a la fois du lexique et de la syntaxe qui, tous deux, cr\'ent des expressions linguistiques hy\'erachiquement structur\'es. 
    	
    	\noindent Une hypoth\`ese concurrente soutient que la capacit\'e g\'en\'erative [du langage] est l'apannage de la syntaxe uniquement (...) la Morphologie Distribu\'ee.
    	
    	\noindent Dans ce mod\`ele, la syntaxe combine des traits abstraits et les prononciations sont ins\'er\'ees pour 'r\'ealiser' ces traits abstraits. Prononciation et interpr\'etation sont compositionnellement extraites de l'arbre syntaxique. Qu'une pi\`ece de structure donn\'ee finisse par \^etre pronc\'ee comme un seul mot d\'epend de l'interaction de la structure syntaxique avec une liste fixe (non g\'e\'erative) qui apparie\footnote{associe} les traits syntaxiques avec des sons et qui est appel\'ee `Vocabulaire'.
    	
    	Toutes les formes diff\'erentes d'un mot r\'esultent de diff\'erentes configurations syntaxiques (...) La morale de cette histoire est : les phrases ne sont pas construites \`a partir des mots -- les mots son construits \`a partir des phrases.
      \end{quote}
      La DM donc repose sur trois piliers :
      \paragraph{Insertion tardive}
        Les exposants (formes) phonologiques des expressions ne sont ins\'er\'ees qu'apr\`es la syntaxe.
      \paragraph{Structure hi\'erarchique jusqu'en bas}
        Les mots sont form\'es de la m\^eme mani\`ere que les syntagmes.
      \paragraph{Sous-sp\'ecification}
        Les exposants phonologiques sont des repr\'esentants (phonologiques) de structures syntaxiques. Il n'est pas n\'ecessaire qu'une forme phonologique donn\'ee fournisse tous les traits syntaxiques sp\'ecifi\'es dans un noeud donn\'e pour \^etre \'eligible comme exposant phonologique de ce noeud. La syntaxe seule fournit les contextes d'apparition des formes phonologiques. 
    \subsubsection{Qu'en est-il du Lexique ?}
        Pour la DM, le Lexique, n'existe la DM rejette cat\'egoriquement deux id\'ees concernant le Lexique 
    \subsubsection{DM et Programme Minimaliste}
    \subsubsection{L'architecture de la Grammaire selon la DM}
    \subsection{Op\'erations post-syntaxiques}
   
    \subsection{Inconv\'enients}
      \subsubsection{Critique de Chomsky 2007}
        L'insertion tardive est violation flagrante de la condition d'inclusion
      \subsubsection{Pas d'op\'erations post-syntaxiques}
        Les op\'erations post-syntaxiques ne peuvent \^etre que des op\'erations phonologiques. Mais \`a partir de ce moment, il y a un probl\`eme avec l'Unification et la Scission : le composant phonologique ne manager les structures syntaxiques. Cela a une autre cons\'equence : l'insertion tardive doit \^etre red\'efinie.
        
  \section{Nano Syntaxe}
    \subsection{Inconv\'enients}
    Que devient SYN et les traits syntaxiques ?
    
    Une une relation directe entre structure syntaxique et sons : inad\'equate pour rendre de changements morphophonologiques. Manque d'une proposition phonologique explicite et satisfaisante. 
    
    Tous ces mod\`eles ignorent dangereusement l'existence d'un composant phonologique. En effet, on voit mal comment combiner ces modeles avec les modeles de phonologie, sans aucune modification, mais voir Lampitelli
  \section{Any solution ?}
    \subsection{Mots et morph\`emes}
      Commen\c cons notre recherche de solution en refl\'echissant sur les questions suivantes : 
      \begin{enumerate}
      	\item[Q1 :] Qu'est-ce qu'un mot ?
      	\item[Q2 :] Qu'est-ce qu'un morph\`eme ?  
      \end{enumerate}
      Nous allons revenir \`a Q1 \`a la fin de cet atelier mais pour l'heure, partons de cette d\'efinition adapt\'ee de Stabler 2014.
      \begin{defin}[Mot]
        X est un mot ssi :
        \begin{enumerate}
      	  \item X est un morph\`eme ou
      	  \item X est un ensemble de mots
        \end{enumerate}
      \end{defin}
      Cette d\'efinition n\'ecessite de r\'epondre \`a Q2. C'est une t\^ache exaltante mais dont on aurait aim\'e s'en passer. En effet, les conceptions structuraliste et ``hallienne'' des morph\`emes n'est pas sans inconv\'enients :
      \begin{itemize}
      	\item \textit{Conception structuraliste} : morph\`emes = atomes s\'emantiques, i.e. plus petite unit\'e significative
      	\item \textit{Conception ``hallienne''} : morph\`emes = atomes syntaxique 
      \end{itemize} 
      Il y a une troisi\`eme conception :
      \begin{quote}
      	``The conclusion drawn by many, if 
      	not most, morphologists is that the
      	morpheme concept is inappropriate except
      	as a crude approximation. Instead, ... morphemes are taken to be
      	markers serving to realize some feature
      	specification ... not listed lexical entries
      	in their own right.'' 
      	
      	(Spencer 2006,
      	p110; cit\'e dans Stabler 2014, page 18.\footnote{voir aussi les ouvrages cit\'es \`a cette page.})
      \end{quote}
      Dans cette conception, vous l'avez remarqu\'e, les morph\`emes \'equivalent \`a des exposants phonologiques dans les entr\'ees de vocabulaire, selon les termes de la DM.
    \subsection{Les formes irr\'eguli\`eres}
      Formes simples ou formes complexes ? voir Stabler lecturenotes 11 - 13., Noyer -- DM allomorphy
      
      Ma proposition : plusieurs sources, possible qu'il y ait eu recombinaison, cf. doublons en fran\c{c}ais.
    
    \subsection{In fine, qu'est-ce qu'un \textit{mot} ?}
         
      L'essentiel de la probl\'ematique de la morphologie s'articule autour de la notion de `mot'. Mais qu'est-ce qu'un `mot' ? \`A la difficult\'e de d\'efinir convenablement le \textit{morph\`eme}, cf. section ..., est attach\'ee celle de d\'efinir convenablement un \textit{mot}. Ce terme est assez ambigu et Ad\'ekpat\'e 2014 ms, dans son cours de morphologie l'a expliqu\'e de fa\c con remarquable. Et m\^eme si nous nous cantonnons sur une conception \`a peu pr\`es `linguistique', la difficult\'e de cerner le `mot' ne disparait pas automatiquement, cf. Stabler 2014, Ms. Il semble que la notion de `mot' ait \'et\'e introduite dans la m\'etalangue pour rendre compte de la \textit{compositionnalit\'e} des expressions linguistiques : les expressions linguistiques sont produire en assemblant des morceaux d'\'enonc\'es pr\'e\'etablis. Cette id\'ee est au centre de l'enseignement du maitre genevois Ferdinand de Saussure [1916, Ms]. Il n'a eu de cesse d'insister sur le fait que le linguiste devait, dans sa t\^ache de description de focaliser sur ces unit\'es isolables. Cela va d'abord prendre la forme de la th\'eorie du signe linguistique -- qui souvent mal comprise d'ailleurs. Mais comme Henri Frei va le montrer quelques ann\'ees plus tard, le signe linguistique peut tout autant correspondre \`a un mot qu'\`a un syntagme. Le concept de signe linguistique ne suffit pas \`a d\'efinir le mot. Poussant la pens\'ee saussurienne jusqu'\`a une limite th\'eorique raisonnable, le structuralisme va proposer la th\`ese de la double articulaticulation du langage : les expressions linguistiques peuvent segment\'ees en unit\'es significatives -- mon\`emes ou morph\`emes selon la terminologie anglosaxonne -- et en phon\`emes si on tient compte de la partie non significative des morph\`emes. Seulement voil\`a, le probl\`eme n'est pas encore r\'esolu puisqu'en \Next, on a bien un seul mot mais deux morph\`emes [k\textsubtilde{\oo}t,Conte'] et [e,Infinitif]. 
      \ex. k\textsubtilde{\oo}te
      
      
      Par ailleurs, les travaux sur la reconnaissance des mots montrent, en l'\'etat actuel des recherches,  que c'est une t\^ache qui mobilise des ressources qui ne sont pas typiquement linguistiques\footnote{D\'ej\`a la reconnaissance des sons que composent un mot d\'epend de facteurs non grammaticaux, cf. Louria ????}. S'il cela est av\'er\'e donc, la reconnaissance des mots serait une t\^ache qui ne regarde pas la FLS, il faudrait alors s'interroger sur la pertinence th\'eorique du concept de \textit{mot} et ses cons\'equnces sur les concepts qui s'y rattachent : lexique, lexical, etc. 
      
      Et au vu de ce qui a \'et\'e dit jusqu'ici, il semblerait que le `mot' n'ait en r\'ealit\'e aucune existence grammaticale. En clair, aucun mot ne fait partie d'un langage-i et encore moins de GU. Or, si les mots ne sont pas des r\'ealit\'es grammaticales, e lexique non n'est pas une r\'ealit\'e grammaticale. C'est une conclusion \`a laquelle nombre de chercheurs travaillant sur la morpholoie dans une perspective minimaliste sont parvenue m\^eme s'ils travaillaient ind\'ependamment. Mais dans ce cas, o\`u se trouvent les mots par rapport \`a ces entit\'es ? 
      
      Pour comprendre cela, commen\c cons par d\'epoussi\'erer une notion aussi vieille que les langage-i mais pr\'esentant des propri\'et\'es qui rappellent \'etrangement la distinction entre FLS et FLL : il s'agit des langages-e. Cette notion nait en m\^eme temps que celle de langage-i [Chomsky 1986, Mufwene 2001] pour d\'esigner en gros tout ce qui dans la langue telle qu'on la con\c coit traditionnelllement, i.e. dans un sens large, ne rel\`eve pas des langage-i. Autrement dit si I est mis pour \textit{Intensionnel} et \textit{Interne}, E est mis pour \textit{Extensionel} et \textit{Externe/Externalis\'e}. Ainsi, les langage-e vont contenir par exemple tout ce qui correspond au style propre \`a un individu -- e.g., le fait qu'il pr\'ef\`ere tel m\'ecanisme ou tel autre --, l'ensemble des productions linguistes d'une communaut\'e -- e.g., proverbes, contes, chants, etc. --  m\^eme les concepts de sociolecte et idiolecte tombent sous la cat\'egorie des langages-e.
      
      Si donc, comme nous l'avons sugg\'er\'e plus haut, le lexique n'appartient pas au langage-i, cela voudrait dire que mots et lexiques composent les langages-e. Mais est-ce \`a dire qu'il ne faut pas les \'etudier ? L\`a n'est pas notre propos.      
      Nous ne disons pas qu'une \'etude lexicale est impossible. Notre propos est que tout d\'epend du niveau auquel on se place. De m\^eme qu'il est possible d'\'etudier l'ethnomu\-si\-co\-logie, l'ethnobothanique ou l'ethnozoologie d'une langue -- terme qui rel\`eve d\'ej\`a de la FLL pour une grande part. Il est possible de parler d'\'etude lexicale tant qu'on se place sur un plan qui n'est pas strictement grammatical [i-langage]. Une telle \'etude serait plus directement rattach\'ee \`a la linguistique appliqu\'ee qu'\'a la linguistique th\'eorique en elle-m\^eme.
      
      Donc ce qu'on va chercher \`a faire dans une analyse lexicale, c'est d\'eterminer les bases linguistiques de l'architecture des \textit{mots} en tant qu'unit\'es mixtes i.e. relevant \`a la fois des e-langages et des i-langages. 
      
%      N\'eanmoins, il ne faudrait pas, il nous semble, \'ecarter toute unit\'e syntaxiquement complexe de l'\'equation. La Nano Syntaxe ainsi que l'approche de Di Sciullo i-morphologie nous appris que des objets proprement linguistiques peuvent \^etre structurellement complexes.
	  
	  N\'eanmoins, m\^eme si le lexique, tel qu'on le con\c coit traditionnellement, tombe, au moins en grande partie, dans le domaine des langages-e, il demeure qu'il nous faut un ensemble d'entr\'ees lexicales dans le sens de Chomsky [1995, et subseq] voir aussi Collins et Stabler 2016. Par cons\'equent, il faudrait distinguer entre lexique-i et lexique-e, le dernier incluant le premier. La question qui se pose maintenant est : quelle est la structure du lexique-i ? quels type d'entr\'ees contient-il ? 
	  
	  Pour r\'epondre \`a cette question, approchons la question par le bas. Quel est le plus petit ensemble d'entr\'ees lexicales, qu'on peut retenir comme puremment grammatical -- parmi celles qui correspondent \`a des ``\textit{mots}'' ou \`a des locutions ?  
	  La premi\`ere premi\`ere r\'eponse qu'on puisse donner \`a cette question est que le lexique-i est un ensemble d'entr\'ees lexicales\footnote{les \textit{mon\`emes} dans le sens de la linguistique structurale}, cf. Collins et Stabler 2016. Cette d\'efinition entraine n\'ecessairement qu'aucun \'el\'ement d'un lexique-i ne peut \^etre structurellement complexe. En clair les bases complexes ne font pas partie des lexiques-i. Toutefois, comme le montrent Abels et al 2007, voir aussi Ramchand 2013 et Aboh 2016, supports de cours pour une id\'ee similaire, les items lexicaux fonctionnels, i.e. morph\`emes grammaticaux dans la terminologie g\'en\'erale, peuvent avoir une structure complexe. En effet, il arrive que des s\'equences phonologiques non composites soient `r\'ealisent' des traits situ\'es \`a diff\'erentes positions, e.g. \traitclass{fin:$ + $}, \traitclass{T:$ + $}, cf. Ramchand 2013.
	  
      \begin{defin}[i-mot/ mot grammatical]
      	X est u i-mot ssi X est un ensemble de \textit{morphes}
      \end{defin}
      \begin{defin}
      	Un \textit{morphe} est un couple \couple{syn,sem}.
      \end{defin}
      Peut-etre est-ce ici le fondement de l'asymetrie entre Phonologie et Semantique -- ou est-ce une autre manifestation de cette asymetrie. 
    \subsection{L'analyse s\'emantique lexicale}
      \subsubsection{Le sens des mots}
        \paragraph{Limites de l'analyse componentielle traditionnelle}
      
        \paragraph{Une approche n\'eo-componentielle}
      
        \paragraph{Sur le changement s\'emantique}
      \subsubsection{Les relations s\'emantiques lexicales}    
    
    \section{Que retenir ?}
      \begin{itemize}
      	\item Il faut distinguer entre Langage au sens Strict qui contient tous les processus purement grammaticaux i.e. propres aux i-langages
      	\item 
      \end{itemize}